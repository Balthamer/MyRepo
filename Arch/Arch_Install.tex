\documentclass[10pt,letterpaper]{article}

\edef\restoreparindent{\parindent=\the\parindent\relax}
\usepackage{parskip}
\usepackage{enumitem}
\restoreparindent
\usepackage[margin=1in]{geometry}
\usepackage{titlesec}
\titleformat{\section}{\large\bfseries}{}{0em}{}

\title{Arch Linux Install}
\author{Michael D. Allred}

\begin{document}

\maketitle

\section{Check for UEFI}

The first task is to ensure that the EFI files are present, this guide will be
for an EFI install. As long as there is output of the ls command then the files
are present then this guide is applicable. If the files are not present, then
the Arch wiki will be the guide to follow.
\begin{itemize}[noitemsep]
    \item ls /sys/firmware/efi/efivars
\end{itemize}

\section{WiFi Connection}

It is preferable to install Arch with a LAN, but if that is not an option then
iwctl can be used to connect to WiFi. The [\,iwd]\# denotes being inside of the
iwctl environment. Inside the environment, use \texttt{wsc list} to show what
wlan interfaces that are able to be used, make note of that interface as it
will be used in the following commands and will replace \textbf{\textit{wlan}}.
With the interface selected, run the \texttt{station wlan get-networks} command
to list the networks that can be connected to as well as the type of
authentication used by that network (\,hidden networks will not show up and if
a hidden network is to be used, skip this step). The name of the network that
will be used will replace \textbf{\textit{network\_name}} and the
authentication type will replace \textbf{\textit{security}}. The most common
type of security will be psk or preshared key, this will be the WiFi password.
The final command for WiFi will be putting the information gathered into one
command and will be responsable for connecting, note that
\textbf{\textit{network\_name}} must be inside quotation.
\begin{itemize}[noitemsep]
    \item iwctl
    \item wsc list
    \item station \textbf{\textit{wlan}} get-networks
    \item station \textbf{\textit{wlan}} connect ``\textbf{\textit{network\_name}}'' \textbf{\textit{security}}
    \item quit
\end{itemize}

\section{Confirm Internet}

The arch linux iso is minimal and only has enough to get users started.
Everything, including the Linux Kernel will need to be downloaded from the
repositories. To make sure that everything is in order use the \texttt{ping}
command to test for internet connectivity. Part of the connectivity is
ensureing the timeing of operations, run the \texttt{timedatectl set-ntp true}
command to set the system clock to the correct time.
\begin{itemize}[noitemsep]
    \item ping archlinux.org
    \item timedatectl set-ntp true
\end{itemize}


\section{Makeing and Mounting EFI Partitions}



\end{document}
